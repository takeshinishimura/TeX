\documentclass[10pt,a4j,uplatex]{jsarticle}
\usepackage[dvipdfmx]{graphicx}
\usepackage[bold,deluxe]{otf}
\usepackage{url}

\pagestyle{empty}

% 余白
\setlength{\topmargin}{-15mm}
\setlength{\headheight}{0mm}
\setlength{\headsep}{0mm}
\setlength{\oddsidemargin}{-7.4mm}
\setlength{\textheight}{277mm}
\setlength{\textwidth}{172mm}
\setlength{\marginparsep}{3mm}
\setlength{\marginparwidth}{50mm}


\newcommand{\testday}{2023年x月xx日(x)}
\newcommand{\testime}{1}
\newcommand{\subject}{授業科目名}
\newcommand{\name}{担当教員名}
\newcommand{\university}{大学名}
\newcommand{\department}{学部名}
\newcommand{\discipline}{}
\newcommand{\grade}{}
\newcommand{\testheader}{%
	\hfill \fbox{試験日}\fbox{\;\testday\testime 時限\;}\\%
	授業科目〔\vbox{\hsize=80pt\centering\subject}〕
	担当教員〔\vbox{\hsize=60pt\centering\name}〕\hfill
	\university\hspace{1em}
	\underline{\vbox{\hsize=50pt\centering\department}}学部\hspace{1em}
	\underline{\hspace{1em}\grade\hspace{1em}}年\\}
\newcommand{\mochikomi}{\underline{※教科書,参考書,配付資料,ノート等持ち込み不可}}
\newenvironment{question}[1]{%
	\noindent\hangindent=4.15em\hangafter=1{\setlength{\fboxrule}{.4pt}\fbox{\mgfamily\textbf{問題 #1}}}\hspace{.5zw}}%
	{\vspace{.1\baselineskip}}
\newenvironment{answer}[1]{%
	\noindent{\setlength{\fboxrule}{.4pt}\fbox{\mgfamily\textbf{問題 #1}}}\par\vspace{.1\baselineskip}}%
	{\vspace{.2\baselineskip}}
\newcommand{\takasa}[1]{\raisebox{-1.5ex}[0pt][0pt]{#1}}

% アイウエオ順
\makeatletter
\def\@aiueo#1{%
	\ifcase#1 □\or ア\or イ\or ウ\or エ\or オ\or カ\or キ\or ク\or ケ\or コ\or
	サ\or シ\or ス\or セ\or ソ\or タ\or チ\or ツ\or テ\or ト\else\@ctrerr\fi%
}
\def\aiueo#1{\expandafter\@aiueo\csname c@#1\endcsname}
\makeatother

\renewcommand{\theenumi}{\mgfamily\ajLabel\ajKuroMaru{enumi}}
\renewcommand{\theenumii}{\mgfamily\aiueo{enumii}}
\renewcommand{\labelenumi}{\theenumi}
\renewcommand{\labelenumii}{\theenumii}

\def\bango#1{\mgfamily\ajKuroMaru{#1}}

\makeatletter
\renewcommand{\p@enumii}{}
\makeatother


\renewcommand{\rmdefault}{ptm}
\renewcommand{\sfdefault}{phv}
%\renewcommand{\familydefault}{\sfdefault}

\def\均等割付#1#2{% 均等割付(#1 文字分,#2 割付対象文字)
	\hbox to #1\bgroup\kanjiskip=0pt plus 1fil minus 1fil #2\egroup%
}


\begin{document}


% 問題用紙
\testheader
{\gtfamily\bfseries 問題用紙}\quad{\small\mochikomi}

\vspace{1\baselineskip}

\begin{question}{1}
次の{\ref{1:first}}から{\ref{1:last}}のそれぞれについて,選択肢の中からもっとも適切なものを選びなさい。(各3点,計18点)

\begin{enumerate}
\setlength{\itemsep}{.5\baselineskip}
	\item ○○の特徴として正しく\.な\.いものは,次のうちどれですか。\label{1:first}
	\begin{enumerate}
		\item foo11
		\item foo12
		\item foo13
		\item foo14\label{1:1}
	\end{enumerate}
	\item ○○の定義として正しいものは,次のうちどれですか。
	\begin{enumerate}
		\item foo21
		\item foo22\label{1:2}
		\item foo23
		\item foo24
	\end{enumerate}
	\item ○○の例として正しく\.な\.いものは,次のうちどれですか。
	\begin{enumerate}
		\item foo31
		\item foo32
		\item foo33\label{1:3}
		\item foo34
	\end{enumerate}
	\item ○○として正しく\.な\.いものは,次のうちどれですか。
	\begin{enumerate}
		\item foo41
		\item foo42\label{1:4}
		\item foo43
		\item foo44
	\end{enumerate}
	\item ○○の説明として正しいものは,次のうちどれですか。
	\begin{enumerate}
		\item foo51
		\item foo52
		\item foo53
		\item foo54\label{1:5}
	\end{enumerate}
	\item ○○として正しく\.な\.いものは,次のうちどれですか。\label{1:last}
	\begin{enumerate}
		\item foo61\label{1:6}
		\item foo62
		\item foo63
		\item foo64
	\end{enumerate}
\end{enumerate}
\end{question}

\vfill

\begin{question}{2}
次の{\ref{2:first}}から{\ref{2:last}}のかっこ内にあてはまる適切な語句または数値を記入しなさい。記入する語句としては,漢字,ひらがな,カタカナのどれでもよい。(各3点,計18点)

\begin{enumerate}
	\item ○○のことを,(\vbox{\hsize=60pt \phantom{}})という。\label{2:first}
	\item ○○することは,(\vbox{\hsize=60pt \phantom{}})という考え方に基づいている。この考え方に○○が,欧米では進められている。
	\item ○○のことを,(\vbox{\hsize=60pt \phantom{}})という。
\vfill
\hfill □\llap{\ajCheckmark}裏面につづく
\newpage
	\item ○○は(\qquad\qquad\qquad )$\textrm{m}^2$である。
	\item ○○のことを,(\vbox{\hsize=60pt \phantom{}})制度という。
	\item 一般的に,○○がよいとされる理由は,○○,すなわち,(\vbox{\hsize=60pt \phantom{}})が小さいことにある。\label{2:last}
\end{enumerate}
\end{question}


\vspace{2\baselineskip}
\begin{question}{3}
次の{\ref{3:first}}から{\ref{3:last}}のそれぞれについて,正しいものに○,誤っているものに×をつけなさい。(各2点,計34点)

\begin{enumerate}
	\item bar1。\label{3:first}
	\item bar2
	\item bar3
	\item bar4
	\item bar5
\vspace{.5\baselineskip}
	\item bar6
	\item bar7
	\item bar8
	\item bar9
	\item bar10
\vspace{.5\baselineskip}
	\item bar11
	\item bar12
	\item bar13
	\item bar14
	\item bar15
\vspace{.5\baselineskip}
	\item bar16
	\item bar17。\label{3:last}
\end{enumerate}
\end{question}

\vspace{2\baselineskip}
\begin{question}{4}
次の{\ref{4:first}}から{\ref{4:last}}のそれぞれの語句を簡潔に説明しなさい。(各5点,計20点)

\newcommand{\語句一}{語句1}
\newcommand{\語句二}{語句2}
\newcommand{\語句三}{語句3}
\newcommand{\語句四}{語句4}

\begin{enumerate}
	\item \語句一\label{4:first}
	\item \語句二
	\item \語句三
	\item \語句四\label{4:last}
\end{enumerate}
\end{question}

\vspace{2\baselineskip}
\begin{question}{5}
次の4つの語句を用いて,○○しなさい。(10点)

{\gtfamily 次の語句を必ず1回は使用すること}:○○,○○,○○,○○
\end{question}
\vfill


% 解答用紙
\newpage
\testheader
{\gtfamily\bfseries 解答用紙}\quad{\small\mochikomi}
\begin{flushleft}
\hfill 学籍番号\underline{\hspace{10zw}}\qquad 氏名\underline{\hspace{10zw}}
\end{flushleft}


\begin{answer}{1}
\noindent
\begin{tabular}{|c|c|c|c|c|c|}
\hline
{\mgfamily\ajKuroMaru{1}} & {\mgfamily\ajKuroMaru{2}} & {\mgfamily\ajKuroMaru{3}} & {\mgfamily\ajKuroMaru{4}} & {\mgfamily\ajKuroMaru{5}} & {\mgfamily\ajKuroMaru{6}}\\
\hline
\takasa{\null} & \takasa{\null} & \takasa{\null} & \takasa{\null} & \takasa{\null} & \takasa{\null}\\
\multicolumn{1}{|p{3zw}}{} & \multicolumn{1}{|p{3zw}}{} & \multicolumn{1}{|p{3zw}}{} & \multicolumn{1}{|p{3zw}}{} & \multicolumn{1}{|p{3zw}}{} & \multicolumn{1}{|p{3zw}|}{}\\
\hline
\end{tabular}
\end{answer}

\vspace{.5\baselineskip}

\begin{answer}{2}
\noindent
\begin{tabular}{|c|p{23.4zw}|c|p{23.4zw}|}
\hline
\takasa{\mgfamily\ajKuroMaru{1}} & \takasa{ } & \takasa{\mgfamily\ajKuroMaru{2}} & \takasa{ }\\
 & & & \\
\hline
\takasa{\mgfamily\ajKuroMaru{3}} & \takasa{ } & \takasa{\mgfamily\ajKuroMaru{4}} & \takasa{ }\\
 & & & \\
\hline
\takasa{\mgfamily\ajKuroMaru{5}} & \takasa{ } & \takasa{\mgfamily\ajKuroMaru{6}} & \takasa{ }\\
 & & & \\
\hline
\end{tabular}
\end{answer}

\vspace{.5\baselineskip}

\begin{answer}{3}
\noindent
\begin{tabular}{|c|c|c|c|c|c|c|c|c|c|}
\hline
{\mgfamily\ajKuroMaru{1}} & {\mgfamily\ajKuroMaru{2}} & {\mgfamily\ajKuroMaru{3}} & {\mgfamily\ajKuroMaru{4}} & {\mgfamily\ajKuroMaru{5}} & {\mgfamily\ajKuroMaru{6}} & {\mgfamily\ajKuroMaru{7}} & {\mgfamily\ajKuroMaru{8}} & {\mgfamily\ajKuroMaru{9}} & {\mgfamily\ajKuroMaru{10}}\\
\hline
\takasa{\null} & \takasa{\null} & \takasa{\null} & \takasa{\null} & \takasa{\null} & \takasa{\null} & \takasa{\null} & \takasa{\null} & \takasa{\null} & \takasa{\null}\\
\multicolumn{1}{|p{3zw}}{} & \multicolumn{1}{|p{3zw}}{} & \multicolumn{1}{|p{3zw}}{} & \multicolumn{1}{|p{3zw}}{} & \multicolumn{1}{|p{3zw}}{} & \multicolumn{1}{|p{3zw}}{} & \multicolumn{1}{|p{3zw}}{} & \multicolumn{1}{|p{3zw}}{} & \multicolumn{1}{|p{3zw}}{} & \multicolumn{1}{|p{3zw}|}{}\\
\hline
{\mgfamily\ajKuroMaru{11}} & {\mgfamily\ajKuroMaru{12}} & {\mgfamily\ajKuroMaru{13}} & {\mgfamily\ajKuroMaru{14}} & {\mgfamily\ajKuroMaru{15}} & {\mgfamily\ajKuroMaru{16}} & {\mgfamily\ajKuroMaru{17}}\\
\cline{1-7}
\takasa{\null} & \takasa{\null} & \takasa{\null} & \takasa{\null} & \takasa{\null} & \takasa{\null} & \takasa{\null}\\
\multicolumn{1}{|p{3zw}}{} & \multicolumn{1}{|p{3zw}}{} & \multicolumn{1}{|p{3zw}}{} & \multicolumn{1}{|p{3zw}}{} & \multicolumn{1}{|p{3zw}}{} & \multicolumn{1}{|p{3zw}}{} & \multicolumn{1}{|p{3zw}|}{}\\
\cline{1-7}
\end{tabular}
\end{answer}

\vspace{.5\baselineskip}

\begin{answer}{4}
\noindent
\begin{tabular}{|c|p{50.35zw}|}
\hline
 & \\
 & \\
\mgfamily\ajKuroMaru{1} & \\
 & \\
 & \\
\hline
 & \\
 & \\
\mgfamily\ajKuroMaru{2} & \\
 & \\
 & \\
\hline
 & \\
 & \\
\mgfamily\ajKuroMaru{3} & \\
 & \\
 & \\
\hline
 & \\
 & \\
\mgfamily\ajKuroMaru{4} & \\
 & \\
 & \\
\hline
\end{tabular}
\end{answer}
\hfill □\llap{\ajCheckmark}裏面につづく

\vspace{.5\baselineskip}

\begin{answer}{5}
\begingroup
\renewcommand{\arraystretch}{46}
\noindent
\begin{tabular}{|p{52.65zw}|}
\hline
\\
\hline
\end{tabular}
\endgroup
\end{answer}


\newpage
\testheader
{\gtfamily\bfseries 解答用紙}\quad{\small\mochikomi}
\begin{flushleft}
\hfill 学籍番号\underline{\hspace{10zw}}\qquad 氏名\underline{\hspace{10zw}}
\end{flushleft}

\begin{answer}{1}
\noindent
\begin{tabular}{|c|c|c|c|c|c|}
\hline
{\mgfamily\ajKuroMaru{1}} & {\mgfamily\ajKuroMaru{2}} & {\mgfamily\ajKuroMaru{3}} & {\mgfamily\ajKuroMaru{4}} & {\mgfamily\ajKuroMaru{5}} & {\mgfamily\ajKuroMaru{6}}\\
\hline
\takasa{\mgfamily\ref{1:1}} & \takasa{\mgfamily\ref{1:2}} & \takasa{\mgfamily\ref{1:3}} & \takasa{\mgfamily\ref{1:4}} & \takasa{\mgfamily\ref{1:5}} & \takasa{\mgfamily\ref{1:6}}\\
\multicolumn{1}{|p{3zw}}{} & \multicolumn{1}{|p{3zw}}{} & \multicolumn{1}{|p{3zw}}{} & \multicolumn{1}{|p{3zw}}{} & \multicolumn{1}{|p{3zw}}{} & \multicolumn{1}{|p{3zw}|}{}\\
\hline
\end{tabular}
\end{answer}

\vspace{.5\baselineskip}

\begin{answer}{2}
\noindent
\begin{tabular}{|c|p{23.4zw}|c|p{23.4zw}|}
\hline
\takasa{\mgfamily\ajKuroMaru{1}} & \takasa{○○} & \takasa{\mgfamily\ajKuroMaru{2}} & \takasa{○○}\\
 & & & \\
\hline
\takasa{\mgfamily\ajKuroMaru{3}} & \takasa{○○} & \takasa{\mgfamily\ajKuroMaru{4}} & \takasa{○○}\\
 & & & \\
\hline
\takasa{\mgfamily\ajKuroMaru{5}} & \takasa{○○} & \takasa{\mgfamily\ajKuroMaru{6}} & \takasa{○○}\\
 & & & \\
\hline
\end{tabular}
\end{answer}

\vspace{.5\baselineskip}

\begin{answer}{3}
\noindent
\begin{tabular}{|c|c|c|c|c|c|c|c|c|c|}
\hline
{\mgfamily\ajKuroMaru{1}} & {\mgfamily\ajKuroMaru{2}} & {\mgfamily\ajKuroMaru{3}} & {\mgfamily\ajKuroMaru{4}} & {\mgfamily\ajKuroMaru{5}} & {\mgfamily\ajKuroMaru{6}} & {\mgfamily\ajKuroMaru{7}} & {\mgfamily\ajKuroMaru{8}} & {\mgfamily\ajKuroMaru{9}} & {\mgfamily\ajKuroMaru{10}}\\
\hline
\takasa{×} & \takasa{×} & \takasa{×} & \takasa{×} & \takasa{×} & \takasa{×} & \takasa{×} & \takasa{○} & \takasa{×} & \takasa{×}\\
\multicolumn{1}{|p{3zw}}{} & \multicolumn{1}{|p{3zw}}{} & \multicolumn{1}{|p{3zw}}{} & \multicolumn{1}{|p{3zw}}{} & \multicolumn{1}{|p{3zw}}{} & \multicolumn{1}{|p{3zw}}{} & \multicolumn{1}{|p{3zw}}{} & \multicolumn{1}{|p{3zw}}{} & \multicolumn{1}{|p{3zw}}{} & \multicolumn{1}{|p{3zw}|}{}\\
\hline
{\mgfamily\ajKuroMaru{11}} & {\mgfamily\ajKuroMaru{12}} & {\mgfamily\ajKuroMaru{13}} & {\mgfamily\ajKuroMaru{14}} & {\mgfamily\ajKuroMaru{15}} & {\mgfamily\ajKuroMaru{16}} & {\mgfamily\ajKuroMaru{17}}\\
\cline{1-7}
\takasa{○} & \takasa{×} & \takasa{×} & \takasa{○} & \takasa{○} & \takasa{×} & \takasa{×}\\
\multicolumn{1}{|p{3zw}}{} & \multicolumn{1}{|p{3zw}}{} & \multicolumn{1}{|p{3zw}}{} & \multicolumn{1}{|p{3zw}}{} & \multicolumn{1}{|p{3zw}}{} & \multicolumn{1}{|p{3zw}}{} & \multicolumn{1}{|p{3zw}|}{}\\
\cline{1-7}
\end{tabular}
\end{answer}

\vspace{.5\baselineskip}

\begin{answer}{4}
\noindent
\begin{tabular}{|c|p{50.35zw}|}
\hline
 & \\
 & \\
\mgfamily\ajKuroMaru{1} & 語句1の説明\\
 & \\
 & \\
\hline
 & \\
 & \\
\mgfamily\ajKuroMaru{2} & 語句2の説明\\
 & \\
 & \\
\hline
 & \\
 & \\
\mgfamily\ajKuroMaru{3} & 語句3の説明\\
 & \\
 & \\
\hline
 & \\
 & \\
\mgfamily\ajKuroMaru{4} & 語句4の説明\\
 & \\
 & \\
\hline
\end{tabular}
\end{answer}
\hfill □\llap{\ajCheckmark}裏面につづく

\vspace{.5\baselineskip}

\begin{answer}{5}
\begingroup
\renewcommand{\arraystretch}{46}
\noindent
\begin{tabular}{|p{52.65zw}|}
\hline
\\
\hline
\end{tabular}
\endgroup
\end{answer}


\end{document}